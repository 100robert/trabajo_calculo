\documentclass[12pt,a4paper]{article}
\usepackage[utf8]{inputenc}
\usepackage[spanish]{babel}
\usepackage{amsmath, amsfonts, amssymb}
\usepackage{graphicx}
\usepackage{xcolor}
\usepackage{geometry}
\usepackage{fancyhdr}
\usepackage{hyperref}
\usepackage{titlesec}
\usepackage{lmodern}
\usepackage{multicol}

% Márgenes
\geometry{left=3cm,right=2.5cm,top=3cm,bottom=3cm}

% Ajuste de la altura del encabezado para fancyhdr
\setlength{\headheight}{27.11232pt}

% Definición de colores sobrios
\definecolor{titlecolor}{RGB}{0, 51, 102}
\definecolor{sectioncolor}{RGB}{0, 0, 0}
\definecolor{Rojo}{RGB}{230, 0, 0}

% Encabezado y pie de página con fancyhdr
\pagestyle{fancy}
\fancyhf{}
\fancyhead[L]{\includegraphics[height=0.8cm]{src/images/logo/logounsch.png}}
\fancyhead[R]{\textcolor{titlecolor}{\dycourse}}

% Pie de página
\fancyfoot[C]{\textcolor{titlecolor}{Universidad Nacional de San Cristóbal de Huamanga}}
\fancyfoot[R]{\thepage}

% Línea en el pie de página
\renewcommand{\footrulewidth}{0.4pt}

% Configuración de columnas
\setlength{\columnsep}{1cm} % Espacio entre columnas
% \setlength{\columnseprule}{0.4pt} % Descomentar para línea divisoria

% Personalización títulos secciones
\titleformat{\section}
  {\color{sectioncolor}\normalfont\large\bfseries}
  {\thesection}{1em}{}
\titleformat{\subsection}
  {\normalfont\normalsize\bfseries}
  {\thesubsection}{1em}{}
% Comandos variables para carátula
\newcommand{\university}[1]{\gdef\dyuniversity{#1}}
\newcommand{\faculty}[1]{\gdef\dyfaculty{#1}}
\newcommand{\dept}[1]{\gdef\dydept{#1}}
\newcommand{\course}[1]{\gdef\dycourse{#1}}
\newcommand{\titleproject}[1]{\gdef\dytitle{#1}}
\newcommand{\tema}[1]{\gdef\dytema{#1}}
\newcommand{\teacher}[1]{\gdef\dyteacher{#1}}
\newcommand{\copyrightyear}[1]{\gdef\dycopyrightyear{#1}}

% Valores variables
\copyrightyear{2025}
\university{Universidad Nacional de San Cristóbal de Huamanga}
\faculty{Facultad de Ingeniería de Minas, Geología y Civil}
\dept{Escuela Profesional de Ingeniería de Sistemas}
\course{CÁLCULO (MA-282)}
\titleproject{Trabajo de Cálculo N°1}
\tema{Resolución de ejercicios de Cálculo Diferencial, Ecuaciones Diferenciables Ordinarias de variable Separable.}
\teacher{Dr. Miyagi}

\begin{document}
\thispagestyle{empty}
\begin{center}
    {\large\scshape \dyuniversity}\\[4pt]
    {\large\scshape \dyfaculty}\\[4pt]
    {\large\scshape \dydept}\\[1cm]
    \includegraphics[width=5cm]{src/images/logo/logounsch.png}\\[1cm]
    {\Large\bfseries \dycourse}\\[0.5cm]
    {\Large\bfseries \dytitle}\\[0.5cm]
    {\large \dytema}\\[2cm]
    \begin{tabular}{rl}
        \textbf{Docente:} & \dyteacher \\
        \textbf{Grupo:} & B \\
        \textbf{Integrantes:} & RAMOS QUINTANILLA, Robert Briceño [27210104] \\
        & VARGAS HUAMAN, Jhoel Alexander [27210105]
    \end{tabular}\\[3cm]
    {\large Ayacucho - Perú}\\[4pt]
    {\large \dycopyrightyear}
\end{center}

\newpage
\setcounter{page}{1}

\begin{multicols}{2}
  

\section*{Ejercicio 1}
\[\tg x \sin^2 y \, dx + \cos^2 xC \tg y \, dy = 0\]
\subsection*{Solución}

\section*{Ejercicio 2}
\[xy\prime -y=y^3\]

\section*{Ejercicio 3}
\[\sqrt{1+x^3}\frac{dy}{dx}=x^2y+x^2 \]

\section*{Ejercicio 4}
\[e^{2x-y}dx+e^{y-2x}dy=0\]

\section*{Ejercicio 5}
\[(x^2y-x^2+y-1)dx+(xy+2y-3y-6)dy=0\]

\section*{Ejercicio 6}
\[e^{x+y}senx \, dx+(2y+1)e^{-y^2}dy=0\]


\section*{Ejercicio 7}
\[3e^x\tg y \, dx+(1-e^x)\sec^2 y \, dy=0\]

rpta: \[\tg y=C(1-e^x)^3\]

\section*{Ejercicio 8}

\[e^y(\frac{dy}{dx}+1)=1\]

rpta: \[\frac{dy}{dx}=\frac{1}{e^y-1}\]
\section*{Ejercicio 9}

\[y\prime=1+x+y^2+xy^2\]

rpta: \[\arctan-x-\frac{x^2}{2}=C\]

\section*{Ejercicio 10}

\[y- xy\prime=a(1+x^2y)\]

rpta: \[y=\frac{a+cx}{1+ax}\]

\end{multicols}

\end{document}