\documentclass[12pt,a4paper]{article}
\usepackage[utf8]{inputenc}
\usepackage[spanish]{babel}
\usepackage{amsmath, amsfonts, amssymb}
\usepackage{graphicx}
\usepackage{xcolor}
\usepackage{geometry}
\usepackage{fancyhdr}
\usepackage{hyperref}
\usepackage{titlesec}
\usepackage{lmodern}




% Márgenes
\geometry{left=3cm,right=2.5cm,top=3cm,bottom=3cm}

% Definición de colores sobrios
\definecolor{titlecolor}{RGB}{0, 51, 102}
\definecolor{sectioncolor}{RGB}{0, 102, 153}

% Encabezado y pie de página con fancyhdr
\pagestyle{fancy}
\fancyhf{}
\fancyhead[L]{\includegraphics[height=0.8cm]{src/images/logo/logounsch.png}}
\fancyhead[C]{\textcolor{titlecolor}{Universidad Nacional de San Cristóbal de Huamanga}}
\fancyhead[R]{\textcolor{titlecolor}{\dycourse}}
\fancyfoot[C]{\thepage}

% Personalización títulos secciones
\titleformat{\section}
  {\color{sectioncolor}\normalfont\Large\bfseries}
  {\thesection}{1em}{}

% Comandos variables para carátula
\newcommand{\university}[1]{\gdef\dyuniversity{#1}}
\newcommand{\faculty}[1]{\gdef\dyfaculty{#1}}
\newcommand{\dept}[1]{\gdef\dydept{#1}}
\newcommand{\course}[1]{\gdef\dycourse{#1}}
\newcommand{\titleproject}[1]{\gdef\dytitle{#1}}
\newcommand{\tema}[1]{\gdef\dytema{#1}}
\newcommand{\teacher}[1]{\gdef\dyteacher{#1}}
\newcommand{\copyrightyear}[1]{\gdef\dycopyrightyear{#1}}

% Valores variables
\copyrightyear{2025}
\university{Universidad Nacional de San Cristóbal de Huamanga}
\faculty{Facultad de Ingeniería de Minas, Geología y Civil}
\dept{Escuela Profesional de Ingeniería de Sistemas}
\course{CÁLCULO (MA-282)}
\titleproject{Trabajo de Cálculo N°1}
\tema{Resolución de ejercicios de Cálculo Diferencial}
\teacher{Dr. Miyagi}

\begin{document}
\thispagestyle{empty}
\begin{center}
    {\large\scshape \dyuniversity}\\[4pt]
    {\large\scshape \dyfaculty}\\[4pt]
    {\large\scshape \dydept}\\[1cm]
    \includegraphics[width=5cm]{src/images/logo/logounsch.png}\\[1cm]
    {\Large\bfseries \dycourse}\\[0.5cm]
    {\Large\bfseries \dytitle}\\[0.5cm]
    {\large \dytema}\\[2cm]
    \begin{tabular}{rl}
        \textbf{Docente:} & \dyteacher \\
        \textbf{Grupo:} & B \\
        \textbf{Integrantes:} & RAMOS QUINTANILLA, Robert Briceño [27210104] \\
        & VARGAS HUAMAN, Jhoel Alexander [27210105]
    \end{tabular}\\[3cm]
    {\large Ayacucho - Perú}\\[4pt]
    {\large \dycopyrightyear}
\end{center}
\newpage
\setcounter{page}{1}

\section{Ejercicio 1}
\textbf{Calcular la derivada de la función:}
\[
f(x) = \frac{x^2}{2}
\]
\textbf{Para los siguientes puntos:}
\[
x_1 = 1, \quad x_2 = 2, \quad x_3 = 3, \quad x_4 = 4
\]
\end{document}
