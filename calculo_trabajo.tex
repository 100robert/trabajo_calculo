\documentclass[12pt,a4paper]{article}
\usepackage[utf8]{inputenc}
\usepackage[spanish]{babel}
\usepackage{amsmath, amsfonts, amssymb}
\usepackage{graphicx}
\usepackage{xcolor}
\usepackage{geometry}
\usepackage{fancyhdr}
\usepackage{hyperref}
\usepackage{titlesec}
\usepackage{lmodern}
\usepackage{multicol}

% Márgenes
\geometry{left=3cm,right=2.5cm,top=3cm,bottom=3cm}

% Ajuste de la altura del encabezado para fancyhdr
\setlength{\headheight}{27.11232pt}

% Definición de colores sobrios
\definecolor{titlecolor}{RGB}{0, 51, 102}
\definecolor{sectioncolor}{RGB}{0, 0, 0}
\definecolor{Rojo}{RGB}{230, 0, 0}

% Encabezado y pie de página con fancyhdr
\pagestyle{fancy}
\fancyhf{}
\fancyhead[L]{\includegraphics[height=0.8cm]{src/images/logo/logounsch.png}}
\fancyhead[R]{\textcolor{titlecolor}{\dycourse}}

% Pie de página
\fancyfoot[C]{\textcolor{titlecolor}{Universidad Nacional de San Cristóbal de Huamanga}}
\fancyfoot[R]{\thepage}

% Línea en el pie de página
\renewcommand{\footrulewidth}{0.4pt}

% Configuración de columnas
\setlength{\columnsep}{1cm} % Espacio entre columnas
\setlength{\columnseprule}{0.4pt} % Línea divisoria entre columnas

% Personalización títulos secciones
\titleformat{\section}
  {\color{sectioncolor}\normalfont\large\bfseries}
  {\thesection}{1em}{}
\titleformat{\subsection}
  {\normalfont\normalsize\bfseries}
  {\thesubsection}{1em}{}
% Comandos variables para carátula
\newcommand{\university}[1]{\gdef\dyuniversity{#1}}
\newcommand{\faculty}[1]{\gdef\dyfaculty{#1}}
\newcommand{\dept}[1]{\gdef\dydept{#1}}
\newcommand{\course}[1]{\gdef\dycourse{#1}}
\newcommand{\titleproject}[1]{\gdef\dytitle{#1}}
\newcommand{\tema}[1]{\gdef\dytema{#1}}
\newcommand{\teacher}[1]{\gdef\dyteacher{#1}}
\newcommand{\copyrightyear}[1]{\gdef\dycopyrightyear{#1}}

% Valores variables
\copyrightyear{2025}
\university{Universidad Nacional de San Cristóbal de Huamanga}
\faculty{Facultad de Ingeniería de Minas, Geología y Civil}
\dept{Escuela Profesional de Ingeniería de Sistemas}
\course{CÁLCULO (MA-282)}
\titleproject{Trabajo de Cálculo N°1}
\tema{Resolución de ejercicios de Cálculo Diferencial, Ecuaciones Diferenciables Ordinarias de variable Separable.}
\teacher{Dr. Miyagi}

\begin{document}
\thispagestyle{empty}
\begin{center}
    {\large\scshape \dyuniversity}\\[4pt]
    {\large\scshape \dyfaculty}\\[4pt]
    {\large\scshape \dydept}\\[1cm]
    \includegraphics[width=5cm]{src/images/logo/logounsch.png}\\[1cm]
    {\Large\bfseries \dycourse}\\[0.5cm]
    {\Large\bfseries \dytitle}\\[0.5cm]
    {\large \dytema}\\[2cm]
    \begin{tabular}{rl}
        \textbf{Docente:} & \dyteacher \\
        \textbf{Grupo:} & B \\
        \textbf{Integrantes:} & RAMOS QUINTANILLA, Robert Briceño [27210104] \\
        & VARGAS HUAMAN, Jhoel Alexander [27210105] \\
        & RUIZ RIVEROS, Roy Jhony [27202123] 
    \end{tabular}\\[3cm]
    {\large Ayacucho - Perú}\\[4pt]
    {\large \dycopyrightyear}
\end{center}

\newpage
\setcounter{page}{1}

\begin{multicols}{2}
  

\section*{Ejercicio 1}
\[\tan x \sin^2 y \, dx + \cos^2 x \cot y \, dy = 0\]
\subsection*{Solución}
Dividimos ambos lados de la ecuación por $\cos^{2}x \, \sin^{2}y$ (suponiendo $\cos x \neq 0$ y $\sin y \neq 0$):
\[
\frac{\tan x \, \sin^{2}y}{\cos^{2}x \, \sin^{2}y} \, dx 
+ 
\frac{\cos^{2}x \, \cot y}{\cos^{2}x \, \sin^{2}y} \, dy = 0
\]
\[
\frac{\tan x}{\cos^{2}x} \, dx + \frac{\cot y}{\sin^{2}y} \, dy = 0
\]

Expresando en términos de senos y cosenos:
\[
\frac{\sin x}{\cos^{3}x} \, dx + \frac{\cos y}{\sin^{3}y} \, dy = 0
\]

Integramos ambos términos:
\[
\int \frac{\sin x}{\cos^{3}x} \, dx + \int \frac{\cos y}{\sin^{3}y} \, dy = C
\]

\noindent
Para la primera integral, sea $u = \cos x \Rightarrow du = -\sin x \, dx$:
\[
\int \frac{\sin x}{\cos^{3}x} \, dx 
= -\int u^{-3} \, du 
= -\left( \frac{u^{-2}}{-2} \right)
\]
\[
= \frac{1}{2}u^{-2}
= \frac{1}{2}\sec^{2}x
\]

\noindent
Para la segunda integral, sea $v = \sin y \Rightarrow dv = \cos y \, dy$:
\[
\int \frac{\cos y}{\sin^{3}y} \, dy
= \int v^{-3} \, dv
= \frac{v^{-2}}{-2}
= -\frac{1}{2}\csc^{2}y
\]

Sustituyendo ambos resultados en la ecuación integral:
\[
\frac{1}{2}\sec^{2}x - \frac{1}{2}\csc^{2}y = C.
\]

Multiplicamos por 2:
\[
\sec^{2}x - \csc^{2}y = C'
\]

Utilizando las identidades trigonométricas:
\[
\sec^{2}x = 1 + \tan^{2}x, 
\qquad 
\csc^{2}y = 1 + \cot^{2}y,
\]
\[
(1+\tan^{2}x) - (1+\cot^{2}y) = C'
\]
\[
\quad \Rightarrow \quad
\tan^{2}x - \cot^{2}y = C'
\]
\[
 - \cot^{2}y = - \tan^{2}x + C'
\]

Múltiplicamos por -1:
\[
\cot^{2}y = \tan^{2}x - C'
\]
Renombramos la constante:
\[
C' = C
\]

\textbf{rpta: $\cot^{2}y = \tan^{2}x + C$}

\section*{Ejercicio 2}
\[xy\prime -y=y^3\]
\subsection*{Solución}
Planteamos la ecuación en forma de derivada:
\[x\frac{dy}{dx} - y = y^3\]

Despejamos la derivada:
\[x\frac{dy}{dx} = y + y^3 = y(1 + y^2)\]

Es una ecuación diferencial \textbf{separable}
\[\frac{dy}{dx} = \frac{y(1 + y^2)}{x}\]

Separamos las variables:
\[\frac{dy}{y(1 + y^2)} = \frac{dx}{x}\]

Descomponemos la fracción en \(y\):
\[\frac{1}{y(1 + y^2)} = \frac{1}{y} - \frac{y}{1 + y^2}\]
ya que 
\[\frac{1}{y} - \frac{y}{1 + y^2} = \frac{1 + y^2 - y^2}{y(1 + y^2)} = \frac{1}{y(1 + y^2)}.\]

Integramos ambos lados:
\[
\int\left(\frac{1}{y} - \frac{y}{1 + y^2}\right)dy = \int \frac{dx}{x}
\]

Calculamos las primitivas:
\[
\int \frac{1}{y}\,dy = \ln|y|, \quad 
\]
\[
\int \frac{y}{1 + y^2}\,dy = \frac{1}{2}\ln(1 + y^2)
\]
(donde se usó la sustitución \(u = 1 + y^2 \Rightarrow du = 2y\,dy\)).

Por tanto:
\[
\ln|y| - \frac{1}{2}\ln(1 + y^2) = \ln|x| + C
\]

Combinando logaritmos:
\[
\ln\left(\frac{|y|}{\sqrt{1 + y^2}}\right) = \ln|x| + C
\]

Exponenciando y absorbiendo constantes:
\[
\frac{|y|}{\sqrt{1 + y^2}} = C_1|x| \quad \Rightarrow \quad \frac{y}{\sqrt{1 + y^2}} = Cx
\]
\[
x = \frac{yC}{\sqrt{1 + y^2}}
\]

\textbf{rpta: $x = \frac{yC}{\sqrt{1 + y^2}}$}

\columnbreak
\section*{Ejercicio 3}
\[\sqrt{1+x^3}\frac{dy}{dx}=x^2y+x^2 \]

\subsection*{Solución}

Planteamos la ecuación en forma de derivada:
\[\sqrt{1+x^3}\frac{dy}{dx}=x^2(y+1)\]

Despejamos la derivada y separamos variables:
\[\frac{dy}{y+1}=\frac{x^2}{\sqrt{1+x^3}}\,dx\]

Integramos ambos lados:
\[
\int\frac{1}{y+1}\,dy=\int\frac{x^2}{\sqrt{1+x^3}}\,dx
\]

Para la integral de la derecha usamos la sustitución
\[u=1+x^3,\qquad du=3x^2\,dx\quad\Rightarrow\quad x^2\,dx=\frac{1}{3}\,du\]

Reemplazando:
\[
\int\frac{1}{y+1}\,dy=\int\frac{1}{3}u^{-1/2}\,du
\]

Calculamos las primitivas:
\[
\int\frac{1}{y+1}\,dy=\ln|y+1|,\qquad
\]
\[
\int\frac{1}{3}u^{-1/2}\,du=\frac{1}{3}\cdot\frac{u^{1/2}}{1/2}=\frac{2}{3}\sqrt{u}
\]

Volviendo a la variable \(x\):
\[
\ln|y+1| = \frac{2}{3}\sqrt{1+x^{3}} + C
\]

Multiplicamos toda la ecuación por $3$ para simplificar:
\[
3\ln|y+1| = 2\sqrt{1+x^{3}} + C'
\]

\textbf{rpta: $\,2\sqrt{1+x^{3}} = 3\ln(y+1) + C\,$}

\columnbreak
\section*{Ejercicio 4}
\[e^{2x-y}dx+e^{y-2x}dy=0\]

\subsection*{Solución}

Dada la ecuación
\[
e^{2x-y}\,dx+e^{y-2x}\,dy=0,
\]
multiplicamos toda la ecuación por \(e^{2x+y}\) para simplificar los exponentes:

\[
e^{2x-y}\cdot e^{2x+y}\,dx + e^{y-2x}\cdot e^{2x+y}\,dy = 0.
\]

Usando la propiedad de exponencial \(e^{a}e^{b}=e^{a+b}\) obtenemos:
\[
e^{(2x-y)+(2x+y)}\,dx + e^{(y-2x)+(2x+y)}\,dy = 0,
\]
es decir
\[
e^{4x}\,dx + e^{2y}\,dy = 0.
\]

Ahora la ecuación es claramente separable:
\[
e^{2y}\,dy = -\,e^{4x}\,dx.
\]

Integramos ambos lados:
\[
\int e^{2y}\,dy = -\int e^{4x}\,dx.
\]

Calculamos las primitivas (usando \(\int e^{ay}\,dy = \frac{1}{a}e^{ay}\)):
\[
\frac{1}{2}e^{2y} = -\frac{1}{4}e^{4x} + C,
\]
multiplicamos por 4 a toda la ecuación:
\[
2e^{2y} + e^{4x} = C_1
\]

\textbf{rpta: $2e^{2y} + e^{4x} = C$}

\columnbreak
\section*{Ejercicio 5}
\[(x^2y-x^2+y-1)dx+(xy+2x-3y-6)dy=0\]

\subsection*{Solución}

Primero factorizamos los grupos que aparecen en los coeficientes:
\[
x^2y-x^2+y-1=(y-1)(x^2+1),
\]
\[
\qquad
xy+2x-3y-6=(y+2)(x-3)
\]
Luego la ecuación queda
\[
(y-1)(x^2+1)\,dx+(y+2)(x-3)\,dy=0
\]

Despejamos \(\dfrac{dy}{dx}\):
\[
(y+2)(x-3)\frac{dy}{dx}=-(y-1)(x^2+1)
\]
\[
\quad\Longrightarrow\quad
\frac{dy}{dx}=-\frac{x^2+1}{x-3}\cdot\frac{y-1}{y+2}
\]

Observamos que la ecuación es \textbf{separable}. Separamos variables:
\[
\frac{y+2}{y-1}\,dy=-\frac{x^2+1}{x-3}\,dx
\]

Integramos ambos lados:
\[
\int\frac{y+2}{y-1}\,dy=-\int\frac{x^2+1}{x-3}\,dx
\]

En la integral de la izquierda descomponemos la fracción:
\[
\frac{y+2}{y-1}=\frac{(y-1)+3}{y-1}=1+\frac{3}{y-1},
\]
\[
\int\frac{y+2}{y-1}\,dy=\int\left(1+\frac{3}{y-1}\right)dy
= y+3\ln|y-1|
\]

En la integral de la derecha hacemos división de polinomios:
\[
\frac{x^2+1}{x-3}=x+3+\frac{10}{x-3},
\]
porque \((x-3)(x+3)=x^2-9\) y \(x^2+1=(x^2-9)+10\). Entonces
\[
-\int\frac{x^2+1}{x-3}\,dx
=-\int\left(x+3+\frac{10}{x-3}\right)dx
\]
\[
=-\left(\frac{x^2}{2}+3x+10\ln|x-3|\right)
\]

Igualando las primitivas y añadiendo la constante de integración \(C\):
\[
y+3\ln|y-1|
\]
\[ = -\left(\frac{x^2}{2}+3x+10\ln|x-3|\right)+C
\]

Reordenando convenientemente obtenemos la solución implícita:
\[
\;y+3\ln|y-1|+\frac{x^2}{2}+3x+10\ln|x-3|=C\
\]
podemos reestructurar la respuesta de esta forma:
\[
\;\ln(y-1)^3+\frac{x^2}{2}+3x+y+\ln(x-3)^{10}=C\
\]

\textbf{rpta: } \(\;\ln(y-1)^3+\frac{x^2}{2}+3x+y+\ln(x-3)^{10}=C\)

\section*{Ejercicio 6}
\[e^{x+y}senx \, dx+(2y+1)e^{-y^2}dy=0\]

\subsection*{Solución}

Dividimos toda la ecuación por \(e^y\) para separar las variables:
\[
e^{x}\sin x\,dx+(2y+1)e^{-y^2-y}\,dy=0
\]

\[
e^{x}\sin x\,dx=-(2y+1)e^{-y^2-y}\,dy
\]

La ecuación es \textbf{separable}. Integramos ambos miembros:
\[
\int e^{x}\sin x\,dx = -\int (2y+1)e^{-y^2-y}\,dy
\]

Calculamos la integral respecto a \(x\). Sea
\[
I=\int e^{x}\sin x\,dx
\]
Usamos integración por partes dos veces:

1. Con \(u=\sin x,\ dv=e^x dx\) \(\Rightarrow\) \(du=\cos x\,dx,\ v=e^x\):
\[
I=e^x\sin x-\int e^x\cos x\,dx
\]

2. Para \(J=\int e^x\cos x\,dx\) hacemos partes con \(u=\cos x,\ dv=e^x dx\) \(\Rightarrow\) \(du=-\sin x\,dx,\ v=e^x\):
\[
J=e^x\cos x+\int e^x\sin x\,dx = e^x\cos x + I
\]

Sustituyendo \(J\) en \(I\):
\[
I = e^x\sin x - (e^x\cos x + I) 
\]

\[\quad\Rightarrow\quad 2I = e^x(\sin x-\cos x)\]

Por tanto
\[
\; \int e^{x}\sin x\,dx = I = \frac{e^x}{2}(\sin x-\cos x)\; 
\]
\noindent
Ahora calculamos la integral respecto a \(y\):
\[
\int (2y+1)e^{-y^2-y}\,dy
\]
\noindent
Para el miembro derecho, hacemos la sustitución:
\[
u = 1 + x^{3} \Rightarrow du = 3x^{2}\,dx
\]
\[
\Rightarrow x^{2}\,dx = \frac{1}{3}\,du.
\]

Sustituyendo en la integral:
\begin{align*}
\int \frac{x^{2}}{\sqrt{1+x^{3}}}\,dx 
&= \frac{1}{3} \int u^{-1/2}\,du \\
&= \frac{1}{3} \cdot 2u^{1/2} \\
&= \frac{2}{3}\sqrt{1+x^{3}} + C_{2}.
\end{align*}
\subsubsection*{Igualando los resultados}
\noindent
Combinando ambas integrales (y absorbiendo las constantes en una sola $C$):
\[
\;\frac{e^x}{2}(\sin x-\cos x)-e^{-y^2-y}=C\;
\]
\noindent
Multiplicamos por 2 a toda la ecuación:
\[
\;{e^x}(\sin x-\cos x)-2e^{-y^2-y}=C\;
\]

\textbf{rpta: $\,2\sqrt{1+x^{3}} = 3\ln(y+1) + C\,$}


\section*{Ejercicio 7} \[3e^x\tg y \, dx+(1-e^x)\sec^2 y \, dy=0\] 
\subsection*{Solución} 
\[3e^x\tg y \, dx+(1-e^x)\sec^2 y \, dy=0\] 

\[3e^x\tg y\,dx=-(1-e^x)\sec^2 y\,dy\] 

\[\frac{3e^x}{1-e^x}=-\frac{\sec^2 y\,dy}{\tg y\,dx}\] 

\[\frac{3e^x}{1-e^x}dx=-\frac{\sec^2 y}{\tg y}dy\] 

\[3\int\frac{e^x}{1-e^x}dx=-\int\frac{\sec^2 y}{\tg y}dy\]  
\noindent
se conose a la funcion $\tg$ y se puede utilizar la regla de la integral de tangente \[\frac{d}{dx}(\tg)=\sec^2x\]
\[\frac{d}{dx}(\tg)=sec^2x\] 
\noindent
sustituyendo los Valores \[u=1-e^x\] \[du=-e^x\,dx\] 

\[-du=e^xdx\]
\noindent
Reemplazando en la integral 

\[3\int\frac{e^x}{1-e^x}dx=\int\frac{\sec^2 y}{\tg y}dy\] 

\[3\int\frac{-du}{u}=-\int\frac{d(\tg y)}{\tg y}dy\] 

\[-3\ln(u)=-\ln(\tg y)+C\] \[\ln(1-e)^3=\ln(\tg y)+C\] 

\[\tg y=C(1-e^x)^3\] 

\textbf{rpta: $\tg y=C(1-e^x)^3$}

\section*{Ejercicio 8}
\subsection*{Solución}
\[e^y(\frac{dy}{dx}+1)=1\]

\[e^y\left(\frac{dy +dx}{dx}\right)=1\]

\[e^ydy+e^ydx=dx\]

\[e^ydy=dx-e^ydx\]

\[e^ydy=dx(1-e^y)\]

\[\frac{e^y}{1-e^y}dy=dx\]

\[\int\frac{e^y}{1-e^y}dy=\int dx\]
\noindent
sustituyendo los Valores \[u=1-e^y\] \[du=-e^y\,dy\] 

\[\int\frac{e^y}{1-e^y}dy=\int\frac{du}{u}\]

\[\ln(u)=\ln(1-e^y)+C\]

\[\ln(1-e^y)=\ln(u)+C\]

\[e^y=u+C\]
rpta: $\frac{dy}{dx}=\frac{1}{e^y-1}$
\section*{Ejercicio 9}

\[y\prime=1+x+y^2+xy^2\]

\subsection*{Solución}
\[y\prime=1+x+y^2+xy^2\]

\[y\prime=(1+x)(1+y^2)\]

\[\frac{dy}{dx}=(1+x)(1+y^2)\]

\[\frac{dy}{1+y^2}=(1+x)dx\]

\[\int\frac{dy}{1+y^2}=\int dx + \int(x)dx\]

\noindent

sabemos que 
\begin{equation*}
\int\frac{du}{a^2+u^2}=\frac{1}{a}\tan^{-1}\left(\frac{u}{a}\right)
\end{equation*}
\noindent
es lo mismo tener $1=1^2 $
\noindent
\textbf{reemplazando}
\[\int\frac{dy}{1^2+y^2}=\int dx + \int(x)dx\]
\noindent
\textbf{usando la regla de la integral de tangente}
\[\int\frac{dy}{1^2+y^2}=\frac{1}{1}\tan^{-1}\left(\frac{y}{1}\right) = \tan^{-1}y\]

\[\tan^{-1}y=x+\frac{x^2}{2}+C\]
\noindent
\textbf{O tambien se puede escribir como}
\[\tan^{-1}y - x - \frac{x^2}{2}=C\]
\noindent
\textbf{se puede escribir como}
\[\arctan y - x - \frac{x^2}{2}=C\]


rpta: $\arctan y-x-\frac{x^2}{2}=C$

\section*{Ejercicio 10}

\[y- xy\prime=a(1+x^2y\prime)\]

\subsection*{Solución}
\[y- xy\prime=a(1+x^2y\prime)\]
\[y= a +ax^2y\prime+xy\prime\]
\[y= a +xy\prime (ax +1)\]
\[xy\prime = \frac{y-a}{ax+1}\]
\[\frac{dy}{dx} = \frac{y-a}{ax^2+x}\]
\[\frac{dy}{y-a}=\frac{dx}{ax^2+x}\]
\noindent
Por lo tanto Integramos ambos lados:
\[\int\frac{dy}{y-a}=\int\frac{dx}{ax^2+x}\]

\[\int\frac{dy}{y-a}=\int\frac{dx}{x^2(a+\frac{1}{x})}\]

\[\int\frac{dy}{y-a}=\int\frac{dx}{x^2}\frac{1}{(a+\frac{1}{x})}\]

\noindent
Sustituimos valores:
\[
u = y - a, \quad du = dy
\]
\[
v = a + \frac{1}{x}, \quad dv = -\frac{1}{x^2}dx \Rightarrow -dv = \frac{1}{x^2}dx
\]

\noindent
Reemplazando en la ecuación integral se tendria:
\[\int\frac{du}{u}=\int-dv\frac{1}{v}\]
\[\int\frac{du}{u}=-\int\frac{dv}{v}\]
\[\ln(u)=-\ln(v)+C_1\]
\[
\ln(u)=\ln\!\left(\frac{1}{v}\right)+C_1
\]
\[
\ln(u)=\ln\!\left(\frac{1}{v}\right)+\ln(K)
\]
\[
\ln(u)=\ln\!\left(\frac{K}{v}\right)
\]
\[
u=\frac{K}{v}
\]
\[
y - a = \frac{K}{a + \frac{1}{x}}
\]
\[
y - a = \frac{Kx}{\frac{ax+1}{x}}
\]
\[
y = a + \frac{Kx}{1+ax}
\]
\[
y = \frac{a(1+ax)+Kx}{1+ax}
\]
\[
y = \frac{a + a^2x + Kx}{1+ax}
\]
\[
y = \frac{a + (a^2+K)\,x}{1+ax}
\]
\[
\text{Sea }C:=a^2+K
\]
\[
\boxed{\,y=\frac{a+Cx}{1+ax}\,}
\]

rpta: \[y=\frac{a+cx}{1+ax}\]
\section*{Ejercicio 11}
\[(1+y^2)dx=(y-\sqrt{1+y^2})(1+x^2)^{\frac{3}{2}}dy\]

\subsection*{Solución}
\[
(1+y^2)\,dx = (y - \sqrt{1+y^2})(1+x^2)^{\frac{3}{2}}\,dy
\]

\noindent
Dividimos ambos lados entre \((1+x^2)^{\frac{3}{2}}(1+y^2)\) para separar variables:
\[
\frac{dx}{(1+x^2)^{\frac{3}{2}}} = \frac{y-\sqrt{1+y^2}}{1+y^2}\,dy
\]

\[
\int\frac{dx}{(1+x^2)^{\frac{3}{2}}}
=
\int\frac{y}{1+y^2}\,dy
-
\int\frac{\sqrt{1+y^2}}{1+y^2}\,dy
\]

---

 1. Resolviendo la primera integral
\[
\int\frac{dx}{(1+x^2)^{\frac{3}{2}}}
\]

\noindent
Basándonos en la fórmula conocida:
\[
\int\frac{du}{(a^2+u^2)^{3/2}} = \frac{u}{a^2\sqrt{a^2+u^2}} + C
\]

\noindent
Aquí \(a=1\), por lo tanto:
\[
\int\frac{dx}{(1+x^2)^{\frac{3}{2}}} = \frac{x}{\sqrt{1+x^2}} + C_1
\]

---

2. Resolviendo la segunda integral
\[
\int\frac{y}{1+y^2}\,dy
\]

\noindent
Sea \(u = 1+y^2 \Rightarrow du = 2y\,dy\):
\[
\int\frac{y}{1+y^2}\,dy = \frac{1}{2}\int\frac{du}{u} = \frac{1}{2}\ln|u| + C_2
\]

\[
\int\frac{y}{1+y^2}\,dy = \frac{1}{2}\ln(1+y^2) + C_2
\]

---

3. Resolviendo la tercera integral
\[
\int\frac{\sqrt{1+y^2}}{1+y^2}\,dy = \int\frac{1}{\sqrt{1+y^2}}\,dy
\]

\noindent
Usando la fórmula conocida:
\[
\int\frac{du}{\sqrt{a^2+u^2}} = \ln|u+\sqrt{a^2+u^2}| + C
\]
con \(a=1\), se tiene:
\[
\int\frac{dy}{\sqrt{1+y^2}} = \ln|y+\sqrt{1+y^2}| + C_3
\]

---

 4. Sustituyendo los resultados en la ecuación
\[
\frac{x}{\sqrt{1+x^2}} = \frac{1}{2}\ln(1+y^2) - \ln|y+\sqrt{1+y^2}| + C
\]

\noindent
Aplicamos la propiedad de logaritmos:
\[
a\ln A - \ln B = \ln\!\left(\frac{A^a}{B}\right)
\]

\[
\frac{x}{\sqrt{1+x^2}} = \ln\!\left(\frac{(1+y^2)^{1/2}}{y+\sqrt{1+y^2}}\right) + C
\]

\[
\boxed{
\ln\left\lvert
\frac{\sqrt{1+y^2}}{y+\sqrt{1+y^2}}
\right\rvert
=
\frac{x}{\sqrt{1+x^2}} + C
}
\]


\section*{Ejercicio 12}
\[(1-y)e^yy\prime+\frac{y^2}{x\ln x}=0\]

\subsection*{Solución}
\[
(1-y)e^{y}\frac{dy}{dx} = -\frac{y^{2}}{x\ln x}
\]
\[
\Rightarrow\quad
\frac{(1-y)e^{y}}{y^{2}}\,dy = -\frac{dx}{x\ln x}
\]

\noindent
Se separan las variables:
\[
\int \frac{(1-y)e^{y}}{y^{2}}\,dy = -\int \frac{dx}{x\ln x}
\]

\noindent
Desarrollamos el numerador:
\[
\int \frac{e^{y}}{y^{2}}\,dy - \int \frac{e^{y}}{y}\,dy = -\int \frac{dx}{x\ln x}
\]

\noindent
Recordemos que:
\[
\frac{d}{dy}\left(-\frac{e^{y}}{y}\right)=e^{y}\frac{1-y}{y^{2}}
\]
\[
\Rightarrow 
\int \frac{(1-y)e^{y}}{y^{2}}\,dy = -\frac{e^{y}}{y} + C_1
\]
\columnbreak

\noindent
Para el otro lado, se usa la fórmula conocida:
\[
\int \frac{dx}{x\ln x} = \ln|\ln x| + C_2
\]

\noindent
Sustituyendo:
\[
-\frac{e^{y}}{y} = -\ln|\ln x| + C
\]

\noindent
Multiplicando por $-1$:
\[
\frac{e^{y}}{y} = \ln|\ln x| + C'
\]

\noindent
Por lo tanto:
\[
\boxed{\,C + \frac{e^{y}}{y} = \ln(\ln x)\,}
\qquad (\text{válido para }x>1,\ y\neq0)
\]

\end{multicols}

\end{document}