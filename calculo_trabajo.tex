\documentclass[12pt,a4paper]{article}
\usepackage[utf8]{inputenc}
\usepackage[spanish]{babel}
\usepackage{amsmath, amsfonts, amssymb}
\usepackage{graphicx}
\usepackage{xcolor}
\usepackage{geometry}
\usepackage{fancyhdr}
\usepackage{hyperref}
\usepackage{titlesec}
\usepackage{lmodern}
\usepackage{multicol}

% Márgenes
\geometry{left=3cm,right=2.5cm,top=3cm,bottom=3cm}

% Ajuste de la altura del encabezado para fancyhdr
\setlength{\headheight}{27.11232pt}

% Definición de colores sobrios
\definecolor{titlecolor}{RGB}{0, 51, 102}
\definecolor{sectioncolor}{RGB}{0, 0, 0}
\definecolor{Rojo}{RGB}{230, 0, 0}

% Encabezado y pie de página con fancyhdr
\pagestyle{fancy}
\fancyhf{}
\fancyhead[L]{\includegraphics[height=0.8cm]{src/images/logo/logounsch.png}}
\fancyhead[R]{\textcolor{titlecolor}{\dycourse}}

% Pie de página
\fancyfoot[C]{\textcolor{titlecolor}{Universidad Nacional de San Cristóbal de Huamanga}}
\fancyfoot[R]{\thepage}

% Línea en el pie de página
\renewcommand{\footrulewidth}{0.4pt}

% Configuración de columnas
\setlength{\columnsep}{1cm} % Espacio entre columnas
\setlength{\columnseprule}{0.4pt} % Línea divisoria entre columnas

% Personalización títulos secciones
\titleformat{\section}
  {\color{sectioncolor}\normalfont\large\bfseries}
  {\thesection}{1em}{}
\titleformat{\subsection}
  {\normalfont\normalsize\bfseries}
  {\thesubsection}{1em}{}
% Comandos variables para carátula
\newcommand{\university}[1]{\gdef\dyuniversity{#1}}
\newcommand{\faculty}[1]{\gdef\dyfaculty{#1}}
\newcommand{\dept}[1]{\gdef\dydept{#1}}
\newcommand{\course}[1]{\gdef\dycourse{#1}}
\newcommand{\titleproject}[1]{\gdef\dytitle{#1}}
\newcommand{\tema}[1]{\gdef\dytema{#1}}
\newcommand{\teacher}[1]{\gdef\dyteacher{#1}}
\newcommand{\copyrightyear}[1]{\gdef\dycopyrightyear{#1}}

% Valores variables
\copyrightyear{2025}
\university{Universidad Nacional de San Cristóbal de Huamanga}
\faculty{Facultad de Ingeniería de Minas, Geología y Civil}
\dept{Escuela Profesional de Ingeniería de Sistemas}
\course{CÁLCULO (MA-282)}
\titleproject{Trabajo de Cálculo N°1}
\tema{Resolución de ejercicios de Cálculo Diferencial, Ecuaciones Diferenciables Ordinarias de variable Separable.}
\teacher{Dr. Miyagi}

\begin{document}
\thispagestyle{empty}
\begin{center}
    {\large\scshape \dyuniversity}\\[4pt]
    {\large\scshape \dyfaculty}\\[4pt]
    {\large\scshape \dydept}\\[1cm]
    \includegraphics[width=5cm]{src/images/logo/logounsch.png}\\[1cm]
    {\Large\bfseries \dycourse}\\[0.5cm]
    {\Large\bfseries \dytitle}\\[0.5cm]
    {\large \dytema}\\[2cm]
    \begin{tabular}{rl}
        \textbf{Docente:} & \dyteacher \\
        \textbf{Grupo:} & B \\
        \textbf{Integrantes:} & RAMOS QUINTANILLA, Robert Briceño [27210104] \\
        & VARGAS HUAMAN, Jhoel Alexander [27210105]
    \end{tabular}\\[3cm]
    {\large Ayacucho - Perú}\\[4pt]
    {\large \dycopyrightyear}
\end{center}

\newpage
\setcounter{page}{1}

\begin{multicols}{2}
  

\section*{Ejercicio 1}
\[\tan x \sin^2 y \, dx + \cos^2 x \cot y \, dy = 0\]
\subsection*{Solución}
Dividimos ambos lados de la ecuación por $\cos^{2}x \, \sin^{2}y$ (suponiendo $\cos x \neq 0$ y $\sin y \neq 0$):
\[
\frac{\tan x \, \sin^{2}y}{\cos^{2}x \, \sin^{2}y} \, dx 
+ 
\frac{\cos^{2}x \, \cot y}{\cos^{2}x \, \sin^{2}y} \, dy = 0,
\]
\[
\frac{\tan x}{\cos^{2}x} \, dx + \frac{\cot y}{\sin^{2}y} \, dy = 0.
\]

Expresando en términos de senos y cosenos:
\[
\frac{\sin x}{\cos^{3}x} \, dx + \frac{\cos y}{\sin^{3}y} \, dy = 0.
\]

Integramos ambos términos:
\[
\int \frac{\sin x}{\cos^{3}x} \, dx + \int \frac{\cos y}{\sin^{3}y} \, dy = C.
\]

\noindent
Para la primera integral, sea $u = \cos x \Rightarrow du = -\sin x \, dx$:
\[
\int \frac{\sin x}{\cos^{3}x} \, dx 
= -\int u^{-3} \, du 
= -\left( \frac{u^{-2}}{-2} \right)
\]
\[
= \frac{1}{2}u^{-2}
= \frac{1}{2}\sec^{2}x.
\]

\noindent
Para la segunda integral, sea $v = \sin y \Rightarrow dv = \cos y \, dy$:
\[
\int \frac{\cos y}{\sin^{3}y} \, dy
= \int v^{-3} \, dv
= \frac{v^{-2}}{-2}
= -\frac{1}{2}\csc^{2}y.
\]

Sustituyendo ambos resultados en la ecuación integral:
\[
\frac{1}{2}\sec^{2}x - \frac{1}{2}\csc^{2}y = C.
\]

Multiplicamos por 2:
\[
\sec^{2}x - \csc^{2}y = C'.
\]

Utilizando las identidades trigonométricas:
\[
\sec^{2}x = 1 + \tan^{2}x, 
\qquad 
\csc^{2}y = 1 + \cot^{2}y,
\]
\[
(1+\tan^{2}x) - (1+\cot^{2}y) = C'
\]
\[
\quad \Rightarrow \quad
\tan^{2}x - \cot^{2}y = C'.
\]
\[
 - \cot^{2}y = - \tan^{2}x + C'.
\]

Múltiplicamos por -1:
\[
\cot^{2}y = \tan^{2}x - C'.
\]
Renombramos la constante:
\[
C' = C.
\]

\textbf{rpta: $\cot^{2}y = \tan^{2}x + C$}

\section*{Ejercicio 2}
\[xy\prime -y=y^3\]

\section*{Ejercicio 3}
\[\sqrt{1+x^3}\frac{dy}{dx}=x^2y+x^2 \]

\section*{Ejercicio 4}
\[e^{2x-y}dx+e^{y-2x}dy=0\]

\section*{Ejercicio 5}
\[(x^2y-x^2+y-1)dx+(xy+2y-3y-6)dy=0\]

\section*{Ejercicio 6}
\[e^{x+y}senx \, dx+(2y+1)e^{-y^2}dy=0\]


\section*{Ejercicio 7} \[3e^x\tg y \, dx+(1-e^x)\sec^2 y \, dy=0\] 
\subsection*{Solución} 
\[3e^x\tg y \, dx+(1-e^x)\sec^2 y \, dy=0\] 

\[3e^x\tg y\,dx=-(1-e^x)\sec^2 y\,dy\] 

\[\frac{3e^x}{1-e^x}=-\frac{\sec^2 y\,dy}{\tg y\,dx}\] 

\[\frac{3e^x}{1-e^x}dx=-\frac{\sec^2 y}{\tg y}dy\] 

\[3\int\frac{e^x}{1-e^x}dx=-\int\frac{\sec^2 y}{\tg y}dy\]  

se conose a la funcion $\tg$ y se puede utilizar la regla de la integral de tangente \[\frac{d}{dx}(\tg)=\sec^2x\]
\[\frac{d}{dx}(\tg)=sec^2x\] 
sustituyendo los Valores \[u=1-e^x\] \[du=-e^x\,dx\] 

\[-du=e^xdx\]

Reemplazando en la integral 

\[3\int\frac{e^x}{1-e^x}dx=\int\frac{\sec^2 y}{\tg y}dy\] 

\[3\int\frac{-du}{u}=-\int\frac{d(\tg y)}{\tg y}dy\] 

\[-3\ln(u)=-\ln(\tg y)+C\] \[\ln(1-e)^3=\ln(\tg y)+C\] 

\[\tg y=C(1-e^x)^3\] 

\textbf{rpta: $\tg y=C(1-e^x)^3$}

\section*{Ejercicio 8}
\subsection*{Solución}
\[e^y(\frac{dy}{dx}+1)=1\]

\[e^y\left(\frac{dy +dx}{dx}\right)=1\]

\[e^ydy+e^ydx=dx\]

\[e^ydy=dx-e^ydx\]

\[e^ydy=dx(1-e^y)\]

\[\frac{e^y}{1-e^y}dy=dx\]

\[\int\frac{e^y}{1-e^y}dy=\int dx\]

sustituyendo los Valores \[u=1-e^y\] \[du=-e^y\,dy\] 

\[\int\frac{e^y}{1-e^y}dy=\int\frac{du}{u}\]

\[\ln(u)=\ln(1-e^y)+C\]

\[\ln(1-e^y)=\ln(u)+C\]

\[e^y=u+C\]
rpta: $\frac{dy}{dx}=\frac{1}{e^y-1}$
\section*{Ejercicio 9}

\[y\prime=1+x+y^2+xy^2\]

\subsection*{Solución}
\[y\prime=1+x+y^2+xy^2\]

\[y\prime=(1+x)(1+y^2)\]

\[\frac{dy}{dx}=(1+x)(1+y^2)\]

\[\frac{dy}{1+y^2}=(1+x)dx\]

\[\int\frac{dy}{1+y^2}=\int dx + \int(x)dx\]



sabemos que 
\begin{equation*}
\int\frac{du}{a^2+u^2}=\frac{1}{a}\tan^{-1}\left(\frac{u}{a}\right)
\end{equation*}

es lo mismo tener $1=1^2 $

\textbf{reemplazando}
\[\int\frac{dy}{1^2+y^2}=\int dx + \int(x)dx\]

\textbf{usando la regla de la integral de tangente}
\[\int\frac{dy}{1^2+y^2}=\frac{1}{1}\tan^{-1}\left(\frac{y}{1}\right) = \tan^{-1}y\]

\[\tan^{-1}y=x+\frac{x^2}{2}+C\]

\textbf{O tambien se puede escribir como}
\[\tan^{-1}y - x - \frac{x^2}{2}=C\]
\textbf{se puede escribir como}
\[\arctan y - x - \frac{x^2}{2}=C\]


rpta: $\arctan y-x-\frac{x^2}{2}=C$

\section*{Ejercicio 10}

\[y- xy\prime=a(1+x^2y)\]

rpta: \[y=\frac{a+cx}{1+ax}\]

\end{multicols}

\end{document}